\documentclass[11pt,a4paper]{article}

% ======================
% Packages
% ======================
\usepackage{graphicx}
\usepackage{amsmath}
\usepackage{physics}
\usepackage{booktabs}
\usepackage{hyperref}
\usepackage{float}
\usepackage{geometry}
\geometry{margin=1in}

% ======================
% Title
% ======================
\title{\textbf{Boosted Decision Tree Analysis for \\ WW Signal vs Top Background}}
\author{Manan Makhija}
\date{\today}

\begin{document}
\maketitle

% ======================
\begin{abstract}
A multivariate analysis using Boosted Decision Trees (BDTs) is performed to
separate the $WW$ signal from dominant top-quark backgrounds using the
TMVA framework. Multiple kinematic and event-level observables are combined
into a single classifier. The performance of the BDT is evaluated using
input variable distributions, correlation matrices, ROC curves, and
overtraining checks. Signal and background efficiencies are reported for
an optimized BDT selection.
\end{abstract}

% ======================
\section{Introduction}

The production of $WW$ events at hadron colliders is an important
electroweak process and constitutes a background to many new physics
searches. A major challenge in isolating the $WW$ signal arises from
large backgrounds originating from top-quark processes, particularly
$t\bar{t}$, $tW$, and $\bar{t}W$ production.

Traditional cut-based analyses often fail to exploit correlations among
kinematic variables. Multivariate techniques such as Boosted Decision
Trees (BDTs) combine multiple observables into a single discriminant and
offer superior separation power. In this work, a BDT-based analysis using
the ROOT TMVA framework is presented.

% ======================
\section{Datasets and Input Variables}

The analysis uses the following Monte Carlo datasets:
\begin{itemize}
    \item $WW$ signal
    \item $t\bar{t}$ background
    \item $tW$ background
    \item $\bar{t}W$ background
\end{itemize}

The BDT is trained using kinematic and event-level variables that are
well motivated by the physics of $WW$ and top-quark production:

\begin{itemize}
    \item Leading and subleading lepton transverse momenta ($p_{T1}, p_{T2}$)
    \item Lepton pseudorapidities ($\eta_1, \eta_2$)
    \item Dilepton invariant mass ($m_{\ell\ell}$)
    \item Dilepton transverse momentum ($p_{T}^{\ell\ell}$)
    \item Azimuthal separation between leptons ($\Delta\phi_{\ell\ell}$)
    \item Missing transverse momentum ($p_T^{\text{miss}}$)
    \item Transverse masses ($m_{T1}, m_{T2}$)
    \item Jet multiplicity ($n_{\text{Jet}}$)
    \item $b$-jet multiplicity ($n_{\text{BJet}}$)
\end{itemize}

% ======================
\section{Input Variable Distributions}

The distributions of the input variables for the training samples are
shown in Figures~\ref{fig:input_vars_1} and~\ref{fig:input_vars_2}. Due to
the large number of variables, the plots are split into two parts.

\begin{figure}[H]
    \centering
    \includegraphics[width=0.9\textwidth]{../plots/input_variable/input_variable_part1.png}
    \caption{Input variable distributions used for BDT training (Part 1).}
    \label{fig:input_vars_1}
\end{figure}

\begin{figure}[H]
    \centering
    \includegraphics[width=0.9\textwidth]{../plots/input_variable/input_variable_part2.png}
    \caption{Input variable distributions used for BDT training (Part 2).}
    \label{fig:input_vars_2}
\end{figure}

Variables related to jet and $b$-jet multiplicities show strong
discrimination between signal and background, while kinematic variables
provide additional separation through correlations exploited by the BDT.

% ======================
\section{Correlation Matrix}

Correlations among the input variables are studied using the linear
correlation matrix provided by TMVA. This allows identification of
redundant variables and ensures that the BDT can effectively exploit
non-linear correlations.

\begin{figure}[H]
    \centering
    \includegraphics[width=0.8\textwidth]{../plots/linear_correlation_coefficients/correlationmatrix_signal.png}
    \caption{Linear correlation matrix of input variables for the training sample (signal).}
    \label{fig:correlation_signal}
\end{figure}

\begin{figure}[H]
    \centering
    \includegraphics[width=0.8\textwidth]{../plots/linear_correlation_coefficients/correlationmatrix_background.png}
    \caption{Linear correlation matrix of input variables for the training sample (background).}
    \label{fig:correlation_background}
\end{figure}

Moderate correlations are observed among kinematic observables, while
jet and $b$-jet multiplicities remain largely uncorrelated with most
leptonic variables.

% ======================
\section{BDT Output Distributions}

\begin{figure}[H]
    \centering
    \includegraphics[width=0.75\textwidth]{../plots/BDT_signal_vs_background.png}
    \caption{Normalized BDT output distribution for $WW$ signal and combined top backgrounds.}
    \label{fig:bdt_overlay}
\end{figure}

Figure~\ref{fig:bdt_overlay} shows the normalized BDT score distribution.
The signal peaks at higher BDT values, while the background is
concentrated at lower values, indicating good separation.

\begin{figure}[H]
    \centering
    \includegraphics[width=0.75\textwidth]{../plots/BDT_stack.png}
    \caption{Stacked BDT output distribution for individual background processes compared with the $WW$ signal.}
    \label{fig:bdt_stack}
\end{figure}

The $t\bar{t}$ process is the dominant background, with smaller
contributions from $tW$ and $\bar{t}W$.

% ======================
\section{BDT Performance}

\subsection{ROC Curve}

The performance of the classifier is quantified using the Receiver
Operating Characteristic (ROC) curve, shown in
Figure~\ref{fig:roc}. The ROC curve illustrates the trade-off between
signal efficiency and background rejection.

The area under the ROC curve (AUC) indicates strong overall classifier performance.

\begin{figure}[H]
    \centering
    \includegraphics[width=0.7\textwidth]{../plots/roc/roc_bdt.png}
    \caption{ROC curve for the BDT classifier.}
    \label{fig:roc}
\end{figure}

\subsection{Overtraining Check}

An overtraining check is performed by comparing the BDT response for
training and testing samples. Good agreement between the two
distributions indicates that the classifier generalizes well.

\begin{figure}[H]
    \centering
    \includegraphics[width=0.7\textwidth]{../plots/overtraining/overtraining_check_bdt.png}
    \caption{Overtraining check for the BDT classifier.}
    \label{fig:overtraining}
\end{figure}

No significant overtraining is observed.

% ======================
\section{Variable Importance}

TMVA provides a ranking of input variables based on their separation
power. The most important variables are found to be the $b$-jet
multiplicity and jet multiplicity, followed by missing transverse
momentum and dilepton kinematics. This ranking is consistent with the
physical expectation that top-quark backgrounds are enriched in
$b$-jets relative to the $WW$ signal.

% ======================
\section{BDT Cut Optimization}

The optimal BDT cut is determined by studying the signal and background
efficiencies as a function of the BDT score. The selected working point
is:

\begin{align}
\text{BDT cut} &= 0.66 \\
\epsilon_S &= 0.68 \\
\epsilon_B &= 0.049
\end{align}

Since the samples are not normalized to physical cross sections, the
analysis focuses on efficiencies rather than absolute signal
significance.

% ======================
\section{Conclusion}

A Boosted Decision Tree analysis using TMVA has been successfully applied
to separate $WW$ signal events from dominant top-quark backgrounds. The
BDT exploits both kinematic and event-level variables and demonstrates
strong discrimination power. The classifier performance has been
validated using input variable studies, correlation matrices, ROC
curves, and overtraining checks. This work highlights the effectiveness
of multivariate techniques in modern high-energy physics analyses.

\end{document}
