\documentclass[11pt,a4paper]{article}

% ======================
% Packages
% ======================
\usepackage{graphicx}
\usepackage{amsmath}
\usepackage{physics}
\usepackage{booktabs}
\usepackage{hyperref}
\usepackage{float}
\usepackage{geometry}
\geometry{margin=1in}

% ======================
% Title
% ======================
\title{\textbf{Boosted Decision Tree Analysis for \\ WW Signal vs Top Background}}
\author{Manan Makhija}
\date{\today}

\begin{document}
\maketitle

% ======================
\begin{abstract}
A multivariate analysis based on Boosted Decision Trees (BDT) is performed to
separate the $WW$ signal from dominant top-quark backgrounds using the TMVA
framework. Signal and background discrimination is achieved using kinematic
and event-level observables. The optimal BDT cut is determined by maximizing
the statistical significance, and the final signal efficiency and background
rejection are reported.
\end{abstract}

% ======================
\section{Introduction}

The production of $WW$ events at hadron colliders constitutes an important
electroweak process and serves as both a signal and a background in many
physics analyses. A major challenge in isolating the $WW$ signal arises from
large backgrounds originating from top-quark processes, particularly
$t\bar{t}$, $tW$, and $\bar{t}W$ production.

Traditional cut-based analyses often fail to exploit correlations among
kinematic variables. Multivariate techniques, such as Boosted Decision Trees
(BDTs), provide superior discrimination power by combining multiple variables
into a single classifier. In this work, a BDT-based analysis using the TMVA
framework is presented.

% ======================
\section{Datasets and Variables}

The analysis is performed using the following datasets:
\begin{itemize}
    \item $WW$ signal
    \item $t\bar{t}$ background
    \item $tW$ background
    \item $\bar{t}W$ background
\end{itemize}

Each dataset is processed to include the BDT output score. The BDT is trained
using the following input variables:

\begin{itemize}
    \item Leading and subleading lepton transverse momenta ($p_{T1}, p_{T2}$)
    \item Lepton pseudorapidities ($\eta_1, \eta_2$)
    \item Dilepton invariant mass ($m_{\ell\ell}$)
    \item Dilepton transverse momentum ($p_{T}^{\ell\ell}$)
    \item Azimuthal separation between leptons ($\Delta\phi_{\ell\ell}$)
    \item Missing transverse momentum ($p_T^{\text{miss}}$)
    \item Transverse masses ($m_{T1}, m_{T2}$)
    \item Jet multiplicity ($n_{\text{Jet}}$)
    \item $b$-jet multiplicity ($n_{\text{BJet}}$)
\end{itemize}

% ======================
\section{BDT Output Distributions}

\begin{figure}[H]
    \centering
    \includegraphics[width=0.75\textwidth]{../plots/BDT_signal_vs_background.png}
    \caption{Normalized BDT output distribution for $WW$ signal and top backgrounds.}
    \label{fig:bdt_overlay}
\end{figure}

Figure~\ref{fig:bdt_overlay} shows the normalized BDT output distributions for
the $WW$ signal and the combined top backgrounds. The signal distribution peaks
at higher BDT values, while the background is concentrated near lower values,
indicating effective separation.

\begin{figure}[H]
    \centering
    \includegraphics[width=0.75\textwidth]{../plots/BDT_stack.png}
    \caption{Stacked BDT output distribution for individual background processes
    compared with the $WW$ signal.}
    \label{fig:bdt_stack}
\end{figure}

Figure~\ref{fig:bdt_stack} shows the stacked background composition compared
with the signal. The $t\bar{t}$ process is the dominant background, with smaller
contributions from $tW$ and $\bar{t}W$.

% ======================
\section{Cut Optimization and Significance}

To determine the optimal BDT selection, the statistical significance
$Z = \frac{S}{\sqrt{S+B}}$ is evaluated as a function of the BDT cut value, where
$S$ and $B$ denote the number of signal and background events, respectively.

\begin{figure}[H]
    \centering
    \includegraphics[width=0.75\textwidth]{../plots/significance_vs_cut.png}
    \caption{Statistical significance as a function of the BDT cut value.}
    \label{fig:significance}
\end{figure}

Figure~\ref{fig:significance} shows the significance as a function of the BDT
cut. The optimal cut is found at:

\begin{align}
    \text{Optimal BDT cut} &= 0.66 \\
    \text{Maximum significance} &= 408.7
\end{align}    

At the optimal cut, the signal and background efficiencies are:
\begin{align}
    \epsilon_S &= 0.681 \\
    \epsilon_B &= 0.049
\end{align}

The final event yields after the BDT selection are:
\begin{align}
    S &= 315{,}144 \\
    B &= 279{,}471
\end{align}

The corresponding Asimov significance is:
\begin{equation}
    Z_A = 517.3
\end{equation}

% ======================
\section{Conclusion}

A Boosted Decision Tree analysis using TMVA has been successfully implemented to
separate the $WW$ signal from dominant top-quark backgrounds. The BDT classifier
provides strong discrimination power, achieving high signal efficiency while
suppressing background contributions. The optimized BDT cut significantly
enhances the statistical significance compared to a cut-based approach. This
study demonstrates the effectiveness of multivariate techniques in modern
high-energy physics analyses.

% ======================
\end{document}
